\begin{resumo}
A programação linear possui aplicações em diversas áreas, o objetivo deste trabalho é explorar uma aplicação na computação. Um modelo de programação linear é utilizado no processo de classificação de dados, que consiste em: para um conjunto de dados com p padrões conhecidos, dado um vetor desse conjunto é possível classificá-lo em um dos p padrões. O método de classificação possui a etapa de treinamento e teste. Na etapa de treinamento, para um par de padrões representados por vetores, que representam pontos no espaço \textit{n}-dimensional, um modelo de programação linear gera um hiperplano que separa os conjuntos de pontos dos dois padrões. Na etapa de teste, utilizando a estrutura de uma árvore binária de torneio e com base nos hiperplanos separadores um vetor é classificado em dos padrões separados pelos hiperplanos. Foram realizados experimentos com quatro conjuntos de dados: dígitos escritos manualmente, gestos de língua brasileira de sinais, espécies da planta Iris e expressões faciais. Obtendo taxas de acerto de classificação variadas para os quatro conjuntos de dados. Através da classificação de dados é possível realizar atividades de reconhecimento como por exemplo, reconhecer uma expressão facial através de uma imagem.

\textbf{Palavras-chave:} Programação linear, classificação de dados, método de classificação.
\end{resumo}
