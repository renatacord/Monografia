\chapter{}

\section{Validação do modelo}
Um modelo de classificação pode ser verificado em relação a sua acurácia apartir de uma conjunto de dados ou a sua performance em relação a outros métodos classificadores podem ser verificados. Algumas técnicas utilizadas com esse intuito serão apresentadas a seguir. No presente trabalho a técnica Cross Validation é utilizada com o objetivo de verificar a acurácia, nesse caso, da metodologia utlizada na classificação de padrões \cite{Kohavi95Cross} \cite{Baldisserotto05Validacao}.

\subsection{Handout}
Esse é um método simples, onde os dados são divididos em 2 grupos mutuamente exclusivos, sendo um grupo utilizado para treino e o outro para teste. Em geral 2/3 dos dados são utilizados no treinamento e 1/3 na etapa de testes. É mais indicado quando uma grande quantidade de dados está disponível, de forma que os dois grupos possam conter dados suficientes de todos os padrões, nesse caso, suficientes para que o resultado final não seja comprometido pela falta de dados no grupo da etapa de treino e nem no grupo de testes. Apesar de ser um método de fácil implementação, a divisão entre conjunto de treinamento e teste deve ser bem estudada de forma que todos os padrões estejam bem representados nos dois grupos, por isso é um método recomendável quando o conjunto de dados possui um grande número de representações de cada padrão \cite{Kohavi95Cross} \cite{Baldisserotto05Validacao}.

\subsection {Cross Validation}
Nesse método, o conjunto de dados também é dividido em cojunto de treinamento e de teste, porém esses dois grupos variam a cada rodada de teste. O tipo de cross validation mais comumente utilizado é o k-fold cross validation, onde os dados são divididos em k conjuntos mutuamente exclusivos e as etapas de treinamento e teste são realizdas K vezes, de forma que a cada rodada um conjunto diferente é utilizado como teste e os outros k-1 conjuntos são utilizados para treinamento. Esse método é recomendável quando o conjunto de dados não é tão grande, pois não existema  preocupação de quais dados separa para teste e quais separa para validação, já que a ao final da execução do método todos o dados terão participado dos conjuntos de validação e teste \cite{Kohavi95Cross} \cite{Baldisserotto05Validacao}.

\subsection{Leave-one-out}
Nessa técnica, do conjunto com o número total de dados N, são realizadas N rodadas de treinamento e teste, sendo que a cada rodada N-1 dados são utlizados para teste e o dados restante é utlizado para treinamento. Pelas suas características, esse é um método recomendável para conjuntos de dados pequenos já que a quantidade de rodadas de treinamento e teste é igual a quantidade de dados o que o torna um método com alto custo computacional \cite{Kohavi95Cross} \cite{Baldisserotto05Validacao}.

TIRAR DO REFERENCIAL TEÓRICO

\subsection{Etapa de validação}

