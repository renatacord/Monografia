\chapter{A Programação Linear e as suas Aplicações}

\section{Introdução}
Na pesquisa operacional, a programação linear é uma das técnicas mais utilizadas em problemas de otimização. Os problemas de programação linear geralmente buscam a distribuição eficiente de recursos limitados para atender um determinado objetivo, por isso suas aplicações estão presentes em diversas áreas como administração, indústria e transporte \cite{Engecom}.

Um problema de programação linear é expresso através de um modelo que é composto por equações e inequações lineares. Esse tipo de problema busca a distribuição eficiente de recursos com restrições para alcançar um objetivo, em geral, maximizar lucros ou minimizar custos. Em um problema de programação linear esse objetivo é expresso através de uma equação linear denominada função objetivo. Para a formulação do problema, é necessário também definir os recursos necessários e em que proporção são requeridos. Essas informações são expressas em equações ou inequações lineares, uma para cada recurso. Esse conjunto de equações ou inequações é denominado restrições do modelo \cite{Engecom}.

\section{Descrição do Problema de Programação Linear}
O modelo de um problema de programação linear normalmente é apresentado em uma das formas a seguir \cite{Passos}:

$\\Max\ z = c^{T}x \\s.a.\left\{\begin{matrix}
Ax\leq b\\x\geq 0 
\end{matrix}\right.$
  ou  $\\Min\ z = c^{T}x \\\\s.a.\left\{\begin{matrix}
Ax\geq  b\\x\geq 0 
\end{matrix}\right.$







