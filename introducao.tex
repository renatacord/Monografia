\chapter{Introdução}

\section{Considerações Iniciais}
A programação linear é uma das disciplinas que compõem a programação matemática e constitui um dos pilares da pesquisa operacional. As aplicações da programação linear estão presentes em diversos setores, tais como nas indústrias, nos transportes, na saúde, na educação, na administração pública. etc.  Além disso, a resolução de problemas de programação linear (PPL) é requerida em outras disciplinas da programação matemática como programação inteira e programação não-linear, onde é comum a resolução de vários PPL de forma repetida.

O método simplex proposto por Danzitg é um dos métodos mais conhecidos e eficientes para resolver problemas de programação linear. Trata-se de um dos poucos algoritmos que foi implantado comercialmente há mais de 40 anos. Atualmente, está presente em softwares comerciais tais como CPLEX e LINGO. Um método alternativo, teoricamente superior ao método simplex, é o método dos pontos interiores, proposto por \citeonline{Karmarkar}. Na prática, tanto o método simplex, quanto o método dos pontos interiores competem até hoje.  

O presente trabalho, por opção, é focado especificamente no método simplex revisado. Busca-se o desenvolvimento de um aplicativo capaz de resolver problemas de programação linear utilizando o método simplex revisado. Acredita-se que a importância desse trabalho se deve ao intuito de explorar o conhecimento a respeito do método, e acima disso o desenvolvimento de uma ferramenta gratuita que proporcione ao usuário uma interface intuitiva e de fácil entendimento.

\section{Objetivos e justificativas}
O objetivo do presente trabalho é o desenvolvimento de um aplicativo para resolver problemas de programação linear utilizando o método simplex revisado. Para o cumprimento do objetivo final o trabalho é composto basicamente por três etapas: 

\begin{itemize}
\item O estudo do método simplex revisado, suas características, vantagens do ponto de vista computacional e suas variantes.
\item O desenvolvimento da biblioteca para resolução de problemas de programação linear
\item O desenvolvimento da interface e do aplicativo em si, incorporando a biblioteca previamente desenvolvida.
\end{itemize}

Através da ferramenta desenvolvida deve ser possível a resolução de problemas de pequeno e médio porte. Além disso, a ferramenta deve oferecer ao usuário uma interface intuitiva e fácil aprendizado.

Apesar de terem sido encontradas ferramentas que resolvem problemas de programação linear durante a fase de pesquisa, este trabalho justifica-se pelo fato de nenhuma dessas ferramentas oferecerem uma interface amigável ou o acesso gratuito ou ambos. Além disso, existe o interesse em desenvolver a capacidade de desenvolver uma biblioteca para resolução de problemas de programação linear com base no método estudado, uma biblioteca open source e que ofereça um bom desempenho e resultados de qualidade.

\section{Estrutura do trabalho}
O presente trabalho apresenta a seguinte estrutura: o capítulo 2 apresenta uma descrição do problema geral de programação linear, os principais métodos de solução, algumas aplicações práticas e as ferramentas computacionais disponíveis. 
