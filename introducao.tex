\chapter{Introdução}

%\section{Considerações Iniciais}
A programação linear é uma das disciplinas que compõem a programação matemática e constitui um dos pilares da pesquisa operacional. As aplicações da programação linear estão presentes em diversos setores, tais como nas indústrias, nos transportes, na saúde, na educação, na administração pública, na computação, etc. Apesar de ser mais comunente utilizada na engenharia de produção, o presente trabalho está focado na aplicação da programação linear na computação gráfica, mais especificamente, no reconhecimento de expressões faciais.

O método simplex proposto por \citeonline{Dan} é um dos métodos mais conhecidos e eficientes para resolver problemas de programação linear. Trata-se de um dos poucos algoritmos que foi implantado comercialmente há mais de 40 anos. Atualmente, está presente em softwares comerciais tais como CPLEX\footnote{http://www-01.ibm.com/software/integration/optimization/cplex-optimizer/} e LINGO\footnote{http://www.lindo.com/}. O método simplex tem como principais carcaterísticas o fato de ser matricial, ou seja, aloca os dados a serem calculados em matrizes, de resolver o conjunto de equações, que formam o modelo de programação linear, de forma interativa até que a solução ótima seja obtida e de ser um método determinístico. Um método alternativo, teoricamente superior ao método simplex, é o método dos pontos interiores, proposto por \citeonline{Karmarkar}. Na prática, tanto o método simplex, quanto o método dos pontos interiores competem até hoje.  

Na área da computação, o reconhecimento de expressões faciais ganhou força no campo da Interação Home-Máquina, onde busca- se uma interação com o computador de forma transparente \cite{Elizabeth}. De acordo com \citeonline{FernandoGil} a detecção facial e o reconhecimento automático de expressões baseia-se no processamento de imagem e reconhecimento de padrões. O processo de reconhecimento da expressão facial é composto, de forma geral, por três etapas \cite{Elizabeth}:\\
\begin{itemize}
\item Detectar a face na cena
\item Extrair as principais características
\item Classificar a imagem em uma detreminada expressão
\end{itemize}

O presente trabalho, por opção, é focado especificamente no método simplex revisado. Busca-se realizar um estudo comparativo da utilização da programação linear no reconhecimento de expressões faciais. Acredita-se que a importância desse trabalho se deve ao intuito de aprimorar o conhecimento a respeito do método, e acima disso explorar a utilização da programação linear na computação gráfica.

\section{Objetivos e justificativas}
O objetivo do presente trabalho é o estudo da utilização da programação linear, mais especificamente do método simplex, no reconhecimento de expressões faciais. 

Através desse estudo deve ser possível verificar a eficiência da programação linear nesse tipo de aplicação da computação gráfica. Além disso, devem ser realizadas duas comparações do resultado final obtido, do ponto de vista computacional: a utilização da programação linear e outro método mais comumente utilizado na computação gráfica; e a utilização de uma biblioteca desenvolvida e uma implementação do modelo de programação linear utlizando o softwrae CPLEX.

A programação linear possui aplicações em diversas áreas, como: indústria, produção, saúde e computação gráfica. Porém é um método mais comumente utilizado na engenharia de produção em problemas como: alocação de recursos e planejamento de produção. O presente trabalho justifica-se pelo fato de querer abordar uma aplicação prática dentro da computação, mais especificamente na computação gráfica, onde a aplicação da programação linear não é tão explorada quanto na engenharia de produção.

\section{Metodologia}
Para o cumprimento do obejtivo final, o trabalho é composto por algumas etapas:

\begin{itemize} 
\item O estudo do método simplex revisado, suas características, vantagens do ponto de vista computacional e suas variantes.
\item O desenvolvimento da biblioteca para resolução de problemas de programação linear
\item A implementação do modelo do problema de programação linear no software CPLEX
\item Aplicação da biblioteca e da implementação feita no CPLEX em alguma aplicação de computação gráfica para reconhecimento de expressões faciais.
\end{itemize}

Para a reralização desta última etapa, espera-se, através de pesquisas encontrar uma aplicação pronta que realize o reconhecimento de expressões faciais que deverá ser utilizada. Porém, ao invés de realizar o processamento utilizando o método já incorporado à aplicação, o método simplex será utilizado. Caso não seja encontrada uma aplicação de reconhecimento de expressão facial, que torne viável a substituição do método utilizado, uma ferramenta de forneça essa funcionalidade será desenvolvida.

\section{Estrutura do trabalho}
O presente trabalho apresenta a seguinte estrutura: o capítulo 2 apresenta uma descrição do problema geral de programação linear, os principais métodos de solução e algumas aplicações práticas; o capítulo 3 apresenta de forma detalhada o Método Simplex Revisado, foco deste trabalho; o capítulo 4 apresenta o processo de reconhecimento de expressões faciais e como a programação linear se aplica neste processo; no capítulo 5 será apresentada a implementação utilizando o software Cplex e a implementação da biblioteca, com as comparações entre os métodos; e no capítulo 6 serão apresentadas as conclusões e possíveis trabalhos futuros. 
