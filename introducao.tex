\chapter{Introdução}

A programação linear é uma das disciplinas que compõem a programação matemática e constitui um dos pilares da pesquisa operacional. As aplicações da programação linear estão presentes em diversos setores, tais como nas indústrias, nos transportes, na saúde, na educação, na computação, etc. Mas é mais comumente aplicada na engenharia de produção, em problemas que buscam a distribuição eficiente de recursos, minimização de gastos e maximização do lucro. O presente trabalho está focado na utilização da programação linear na computação, mais especificamente no processo de classificação de dados em padrões. 

Um dos focos dos estudos das aplicações da programação linear é na sua utilização na separação de pontos no espaço \textit{n}-dimensional. Na computação esse tipo de aplicação pode ser utilizada em atividades com o objetivo de classificar dados, como por exemplo:

\begin{itemize}
\item Dígitos escritos manualmente 
\item Expressões faciais
\item Espécies de plantas
\item Gestos manuais
\end{itemize}

De forma geral, esse tipo de aplicação da programação linear pode ser utilizado em abordagens que permitam a representação dos dados em forma de vetor numérico. Esse vetor de tamanho n representa um ponto no espaço n-dimensional e cada valor do vetor representa uma característica, por isso é denominado vetor de características.

No processo de classificação de dados, os padrões são as possíveis classificações que um dado pode receber. 

No presente trabalho é utilizada a metodologia para classificação de dados apresentada em \citeonline{Feng} e \citeonline{Guo}. Primeiramente, o modelo de programação linear (Programação Linear de Hiperplano Separador (MPLHS)) proposto por \citeonline{Bennett92robustlinear} é utilizado na geração de hiperplanos, cada hiperplano separa dois conjuntos de pontos, onde cada conjunto representa um padrão. Na etapa de classificação de um dado, com padrão inicialmente desconhecido, uma árvore binária de torneio é utilizada na determinação de qual padrão o dado pertence com base nos hiperplanos gerados.

%Utiliza-se um modelo de programação linear que gera um hiperplano no espaço n-dimensional que separa %dois conjuntos de pontos, cada conjunto representando um padrão, posteriormente esse hiperplano é %utilizado na classificação de um dado em um desses dois padrões. 

%O modelo de programação linear adotado separa
O MPLHS dois conjuntos de pontos, porém também pode ser utilizado em problemas onde busca-se a separação de múltiplos padrões. O presente trabalho foca nessa última abordagem, em todos os testes realizados o número de conjuntos de pontos (padrões) é igual ou maior que três.

%para a resolução do modelo de programação linear
O método simplex proposto por \citeonline{Dan} é utilizado para a resolução do MPLHS. É um dos métodos mais conhecidos e eficientes para resolver problemas de programação linear. Foi implantado comercialmente há mais de 40 anos e atualmente está presente em softwares comerciais tais como CPLEX e LINGO. 


\section{Objetivos e justificativas}
O objetivo do presente trabalho é o estudo da utilização da programação linear e do método simplex, na classificação de dados em padrões. O MPLHS gera hiperplanos, cada hiperplano separa dois padrões representados por dois conjuntos de pontos e uma árvore binária de torneio realiza a classificação de um dado com padrão inicialmente desconhecido.

Através desse estudo deve ser possível verificar a eficiência da programação linear nesse tipo de aplicação. São realizados testes com quatro conjuntos de dados para verificar a eficácia da metodologia utilizada que é composta pelo MPLHS na separação de padrões e uma árvore de torneio para classificação. Três desses conjuntos foram obtidos já na forma de vetores de características, o quarto conjunto foi obtido em forma de imagens e um método de extração de características foi utilizado para obter os vetores.

A programação linear possui aplicações em diversas áreas, como: indústria, produção, saúde e computação gráfica. Porém é um método mais comumente utilizado na engenharia de produção. O presente trabalho justifica-se pelo fato de abordar uma aplicação prática dentro da computação, onde a aplicação da programação linear não é tão explorada quanto na engenharia de produção.

\section{Metodologia}
Para o cumprimento do objetivo final, o trabalho é composto pelas seguintes etapas:

\begin{itemize} 
\item O estudo do método simplex revisado, suas características e vantagens do ponto de vista computacional;
\item Obtenção de dados para testes;
\item A implementação do  MPLHS utilizando a linguagem de programação JAVA juntamente com o software CPLEX;
\item A implementação da etapa de classificação através da árvore binária de torneio;
\item Realização de testes e análise dos resultados.
\end{itemize}

\section{Estrutura do trabalho}
Este trabalho está estruturado em seis capítulos. O primeiro capítulo apresenta a introdução ao tema abordado, os objetivos e justificativas e a metodologia descrevendo as etapas para o cumprimento do trabalho.

O segundo capítulo é composto pela descrição geral do problema de programação linear, algumas aplicações práticas, uma apresentação de alguns métodos de classificação e métodos de validação da metodologia, além de alguns trabalhos relacionados à classificação de dados.

O capítulo 3 apresenta de forma detalhada o Método Simplex e outro importante método de solução de problemas de programação linear.

No quarto capítulo é apresentado o MPLHS utilizado na geração dos hiperplanos separadores e o processo de classificação de dados.

O capítulo 5 apresentada os experimentos realizados, suas análises e os dados utilizados

O sexto e último capítulo trata das considerações finais, conclusões e propostas de continuação deste trabalho.
