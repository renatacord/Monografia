\section{Considerações Iniciais}
A programação linear é uma das disciplinas que compõem a programação matemática e constitui um dos pilares da pesquisa operacional. As aplicações da programação linear estão presentes em diversos setores, tais como nas indústrias, nos transportes, na saúde, na educação, na administração pública, etc. Além disso, a resolução de problemas de programação linear (PPL) é requerida em outras disciplinas da programação matemática como programação inteira e programação não-linear, onde é comum a resolução de vários PPL de forma repetida.

O método simplex proposto por Dantzig (1947) é um dos métodos mais conhecidos e eficientes para resolver problemas de programação linear. Trata-se de um dos poucos algoritmos que foi implantado comercialmente há mais de 40 anos. Atualmente, está presente em softwares comerciais tais como CPLEX, XPRESS e LINGO. Um método alternativo, teoricamente superior ao método simplex, é o método dos pontos interiores, proposto por Karmarkar (1984). Na pratica, tanto o método simplex, quanto o método dos pontos interiores competem até hoje.  

O presente trabalho, por opção, é focado especificamente no método simplex revisado. Busca-se o desenvolvimento de um aplicativo capaz de resolver problemas de programação linear utilizando o método simplex revisado. Acredita-se que a importância desse trabalho se deve ao intuito de explorar o conhecimento a respeito do método, e acima disso o desenvolvimento de uma ferramenta gratuita que proporcione ao usuário uma interface intuitiva e de fácil entendimento.
