\chapter{Conclusão}

A programação linear, apesar de possuir aplicações em diversas áreas, não é tão explorada na área da computação. O presente trabalho aprsentou a utilização de uma modelo de programação lienar no reconhecimento de padrões. Como resultado, esse modelo gera hiperplanos separadores que são utlizados na tarefa de classificar um dado com padrão inicialmente desconhecido.

Inicialmente foi realizado um estudo sobre aplicações de programação linear e sobre métodos de resolução, princiaplmente sobre o método simplex revisado, um método desenvolvido para problemas de programação linear resolvidos computacionalmente. Também foram aprsentados alguns metódos de classificação de padrões e métodos de validação da metodologia.

Foram realizados experimentos utilizando quatro conjuntos de dados, três desses conjuntos foram obtidos já no formato de vetores de carcaterísticas. No caso do conjunto de dados Expressões, foram obtidas imagens e os vetores de características foram extraídos utilizando o método Local Binary Patterns. Nos experimentos, o modelo de programação linear proposto por \citeonline{Bennett92robustlinear} foi utlizado, os padrões foram combinados em pares e os vetores de cada padrão submetidos ao modelos, gerando um hiperplano separador para cada combinação. Já na etapa de classificação foi utlizada uma estrtura de árvore binária de torneio, o vetor a ser classificado era subemtido a árvore e um padrão era retornado classificando esse vetor. Para a validação do modelo o método k -fold cross valdiation foi utlizado devido a variação de tamanho das bases de dados utilizadas. Na implementação foi utlizada a linguagem de programação JAVA e o software CPLEX da IBM.

Nos experimentos foram obtidas taxas de acerto altas para os casos da base de dados Dígitos e Íris, com taxas de acerto 98\% e 85\%, respectivamente. No caso da base de dados LIBRAS foi obtida uma taxa de acertos de 74\%. Para a base de dados Expressões a taxa obtida foi de 36\%, esse resultado pode ser atribuído ao vetor de caracteríticas, já que os vetores foram extraídos sem um pre processamento das imagens. Utilizando a base de dados Dígitos foi analisada a influência da quantidade de dados na melhoria da taxa de acertos, foi verificado que essa taxa pode aumentar a medida que a quantidade de dados por padrão aumenta, porém após uma certa quantidade de vetores por classe, o aumento na taxa de acerto não é tão significante. Já no teste para análise do tempo computacional foi constatado que a quantidade de padrões e de vetores por padrões influencia no tempo, porém a quantidade de vetores tem uma maior influencia no aumento do tempo computacional.

O presente trabalho expôs uma aplicação na programação linear na computação, mais especificamente na classificação de padrões. Essas aplicações contribuem na interação mais transparente homem - máquina.
Foram obtidos bons resultados em alguns experimentos, e em alguns casos, superiores ao resultados obtidos em outros trabalhos que utilizaram a mesma base de dados e um método de classificação diferente. Em outros experimentos os resultados obtidos não foram tão satisfatórios.

Como trabalho futuro propõe-se a melhoria na taxa de acertos, principalmente da base de dados Expressões. Para isso deve haver um estudo amis aprofundado na área da visão computacional, submetendo as imagens a um pre processamento e utilizando uma método mais apurado para selecionar as características e reduzir o tamanho do vetor de características.

Um outra melhoria é o tempo computacional, que poderia ocorrer através de um estudo sobre as melhores estrturas de dados a serem utlizadas.   
